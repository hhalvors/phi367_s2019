\documentclass[fleqn,12pt]{article}
\sloppy
% \usepackage{fullpage}
\usepackage{outlines}
\usepackage{enumitem} 
\setlength{\parskip}{1em}
\setlength{\parindent}{0em}
\usepackage{setspace}
% \spacing{1.35}
\begin{document}

\begin{enumerate}
\item (p 300) What do you think that K means by ``the language of
  abstraction''?  In what contexts might it be appropriate to speak in
  the language of abstraction?  (See also p 314)
\item What features of abstract thinking prevent it from capturing the
  situation of ``existing individuals''?
\item (p 303--304) There seems to be a critique here of two kinds of
  people: ``a thinker'' and ``an artist''.  What features of their
  lives bother the writer?
\item (p 305) K says that Hegel is perfectly right in maintaining that
  there is no either/or in the language of abstraction, and that
  abstraction removes contradictions.  What in the world does he mean
  by that?
  \begin{enumerate}
  \item What is it about existence that demands an either/or? (See
    also p 306 where he says ``And if he exists, is he not in the
    process \dots '')
  \end{enumerate}
\item (pp 306--307) What role does the \emph{future} play in this
  discussion?
\item (p 308) How does \emph{motion} relate to existence?  Why might K
  think that motion cannot be described in the language of
  abstraction?
\item (p 308) What does it mean to say that ``all thinking is
  eternal''?  And why is it supposed to follow from that that
  existence cannot be captured in thought?
\item (p 309) What might K say against the claim that ``science can
  explain everything''?
  \begin{enumerate}
  \item (p 310) What ultimately is the problem with Hegelian
    philosophy?
  \item (p 310) What role does the \emph{past} play within Hegelian
    philosophy?
  \item (p 310) How much can pure thinking help with existing?  
  \end{enumerate}
\item (p 312) There seems to be a description here of the ``task'' of
  an existing person.  Explain what the task is.  (Hint: it's not
  simply existing, and it's not simply moving.)
\item (p 313) There's a difficult passage here about ``pure thinking''
  versus ``abstract thinking''.  What in the world is he talking about
  here?  Is there any relation to Kant's various types of thinking?
\item (p 314) What does abstraction do vis-a-vis actuality and
  possibility?
\item (p 314) Before looking at the text, how would you define the
  relation between possibility and actuality?  Now, at the bottom of
  page 314, what does K say about the relation between thinking,
  being, actuality, and abstraction?
\item (p 314) How does ethics relate to the principle of
  contradiction?
\item (p 316) What does K mean by ``all knowledge about actuality is
  possibility'' and ``genuine knowledge is a translation into
  possibility'' and ``with knowledge he is in the medium of
  possibility''?
\item (p 316, bottom) How does K turn Descartes' {\it Cogito} argument
  on its head?
  

\end{enumerate}


\end{document}


%%% Local Variables:
%%% mode: latex
%%% TeX-master: t
%%% End: